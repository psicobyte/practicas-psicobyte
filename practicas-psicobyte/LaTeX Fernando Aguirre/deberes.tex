\documentclass[a4paper,openright,12pt]{book} 
\usepackage[spanish]{babel} 
\usepackage[latin1]{inputenc}
\usepackage{graphicx}
\begin{document}
\begin{titlepage} 


%\vspace*{-1in}
\begin{figure}[htb]
\begin{center}
\includegraphics[width=8cm]{./figuras/dawrin.jpg}
\end{center}
\end{figure}
\begin{center} 
\begin{Huge} 
\textsc{Insectos de Sierra Nevada} 
\end{Huge} 
\end{center} 
\end{titlepage}
\newpage
$\ $
\thispagestyle{empty}
\chapter*{}
\pagenumbering{Roman} 
\begin{flushright}
\textit{Dedicado a \\
Darwin Eventur}
\end{flushright}
\newpage
\tableofcontents
\newpage
\listoffigures
\chapter*{Resumen}
%\addcontentsline{toc}{section} 
%\markboth{RESUMEN}{RESUMEN}
Coleoptera es un Orden de insectos tambi\'{e}n conocidos como escarabajos gorgojos o picudos. Dentro del Parque Natural de Sierra Nevada encontramos unas adaptaciones muy interesantes que destacan respecto al resto de la regi\'{o}n. \\ En este trabajo se hablar\'{a} sobre algunas de ellas y su relaci\'{o}n con el medio

%\newpage
\chapter{Cole\'{o}pteros}
A este orden pertenecen los insectos conocidos como escarabajos, gorgojos o picudos. La mayor parte de los individuos de las especies de este orden que hacen de Sierra Nevada su alimento y su refugio y su hogar presentan un color pardo u oscuro (Figura 1.1). Esta caracter\'{i}stica es una adaptaci\'{o}n que les proporciona dos ventajas.\\ la primera es el color negro absorbe la energ\'{i}a luminosa del sol al ser un color que absorbe en todas las longitudes de onda del espectro de luz visible por lo que los individuos negros u oscuros tienen m\'{a}s temperatura que los claros. Esto es una ventaja fundamental en zonas de alta montana y climas fr\'{i}os. \\ Para entender la segunda ventaja que les proporciona el color oscuro a los cole\'{o}pteros primero hay que saber que el sustrato de sierra nevada en zonas altas est\'{a} compuesto fundamentalmente de esquistos como la pizarra y en zonas m\'{a}s bajas de calizas y dolom\'{i}as, en cualquier caso es oscuro por lo que un color oscuro les da una capacidad mim\'{e}tica eficaz para pasar desapercibidos y no ser capturados por ning\'{u}n depredador o entom\'{o}logo malicioso. \\
\begin{figure}
\begin{center}
\includegraphics[scale=1]{./figuras/bicho.jpg}
\caption{\footnotesize Adulto de \textit{Pimelia baetica}}
\label{figura 1.1}
\end{center}
\end{figure}
El desarrollo de estos insectos es un ciclo completo en el que están presentes diferentes estadios de desarrollo. El primero es el huevo, la hembra mediante el uso de su ovopositor suele poner un gran n\'{u}mero de huevos de color claro usualmente en zonas axilares de algunas especies de plantas, en huecos que encuentran en troncos de madera en descomposici\'{o}n o incluso debajo de las piedras.\\ El segundo estadio es el larvario, el escarabajo es una larva comúnmente llamada ``gusano'' que no tiene nada que ver con el grupo de los an\'{e}lidos que son los gusanos propiamente dichos, estas larvas pueden ser de diferentes formas y colores desde la peluda \textit{Dermestes maculatus} (Figura 1.2) hasta la p\'{a}lida \textit{Lucanus cervus} pasando por la curiosa \textit{Coccinella septempunctata} y sus tiempos de desarrollo en algunos casos como en el caso de \textit{D. maculatus} o \textit{N. vespillo} ayudan en diagn\'{o}sticos forenses a esclarecer la data de la muerte. \\ El tercero es el de pupa o cris\'{a}lida, durante este estadio que puede durar algunas semanas la larva de cole\'{o}ptera sufre una metamorfosis que dar\'{a} lugar a un adulto.
\begin{figure}
\begin{center}
\includegraphics[scale=.5]{./figuras/vicho.jpg}
\caption{\footnotesize Larva de \textit{Dermestes maculatus}}
\label{figura 1.2}
\end{center}
\end{figure}
El \'{u}ltimo estad\'{i}o es el de adulto, cerrando de este modo el ciclo, los adultos presentan diversas formas, colores y estructuras que varian en funci\'{o}n de las presiones selectivas a las que han sido sometidos a lo largo de su historia evolutiva, antes comentado este tema para los que encontramos en Sierra Nevada
\chapter{Neur\'{o}pteros}
Para hablar de este grupo de insectos me voy a basar en las hormigas le\'{o}n y las crisopas.\\
En este orden encontramos insectos en cuyo tambi\'{e}n tienen cuatro fases, la fase huevo, la fase juvenil o de larva, la fase de cris\'{a}lida y la fase de adulto. Los adultos (figura 2.1) poseen dos pares de alas membranosas con gran cantidad de venas y v\'{e}nulas, este patr\'{o}n tan abundante de venas y v\'{e}nulas en las alas da nombre al orden por parecerse a las 	dendritas de las c\'{e}lulas nerviosas. \\ 

Los adultos tienen unas mand\'{i}bulas fuertes que les permiten desbrozar las duras cut\'{i}culas de los insectos que cazan para comer, principalmente insectos del orden ``d\'{i}ptera'' (moscas diminutas y mosquitos) que cazan al vuelo. Los agarran con su primer par de patas en el que tienen unas diminutas esp\'{i}culas, cuyo n\'{u}mero puede variar en funci\'{o}n de cada especie y que dificultan que sus presas se escapen. \\
\begin{figure}
\begin{center}
\includegraphics[scale=.15]{./figuras/adulta.jpg}
\label{figura 2.1}
\caption{\footnotesize Neur\'{o}ptero adulto}
\end{center}
\end{figure} \\
Las larvas o juveniles (figura 2.2) constituyen el principal atractivo de este orden de insectos, estas 	son pequenas y de forma ovalada y aplanada con dos grandes mand\'{i}bulas que pueden igualar el tamano del propio cuerpo de la ninfa. 
\begin{figure}
\begin{center}
\includegraphics[scale=.25]{./figuras/neuroptera.jpg} 
\caption{\footnotesize  Larva de hormiga le\'{o}n}
\label{figura 2.2}
\end{center}
\end{figure}

\end{document}
